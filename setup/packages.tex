% Load packages according to your needs

\usepackage{amsmath}
\newtheorem{theorem}{定理}
\newtheorem{lemma}{引理}
\newtheorem{proof}{证明}[subsection]
\usepackage{amssymb}
\usepackage[noend]{algpseudocode}
\usepackage{algorithm}
\usepackage{algorithmicx}
\floatname{algorithm}{算法}
\usepackage{array}
\usepackage{multirow}

\usepackage{fancyhdr}
\pagestyle{fancy}
\lhead{}
\chead{\leftmark}
\rhead{}
\renewcommand{\headrulewidth}{0.4pt}
\renewcommand{\footrulewidth}{0.4pt}

\usepackage{subcaption}
\usepackage{graphicx}
% 设置插图目录
\graphicspath{{./figs/}}

\usepackage{xcolor}
\definecolor{wheat}{RGB}{245,222,179}   

\setCJKfamilyfont{\CJKsfdefault}{Times New Roman}
\newcommand\timesnewroman{\CJKfamily{\CJKsfdefault}}
\setCJKfamilyfont{\CJKsfdefault}{SIMFANG.TTF}
\renewcommand\fangsong{\CJKfamily{\CJKsfdefault}}
%\setCJKfamilyfont{\CJKrmdefault}{SimSun}
%\renewcommand\songti{\CJKfamily{\CJKrmdefault}}
\setCJKfamilyfont{\CJKsfdefault}{SimHei}
\renewcommand\heiti{\CJKfamily{\CJKsfdefault}}
\usepackage{biblatex}
\usepackage{hyperref}
\hypersetup{
	colorlinks=true,
	linkcolor=blue,
	filecolor=blue,      
	urlcolor=blue,
	citecolor=cyan,
}
\def\equationautorefname{\textbf{公式}}%
\def\algorithmautorefname{\textbf{算法}}% 
\def\tableautorefname{\textbf{表}}%
\def\figureautorefname{\textbf{图}}%
\def\lemmaautorefname{\textbf{引理}}%
\def\theoremautorefname{\textbf{定理}}%
\usepackage{titlesec}
\titleformat{\section}{\centering\LARGE\heiti}{第\,\thesection\,章}{1em}{}
\titleformat{\subsection}{\centering\large\heiti}{第\,\thesubsection\,节}{1em}{}
\titleformat{\subsubsection}{\centering\small\heiti}{\thesubsubsection}{1em}{}
\usepackage{minted}
\usepackage{listings}
\usepackage{color}
\definecolor{dkgreen}{rgb}{0,0.6,0}
\definecolor{gray}{rgb}{0.5,0.5,0.5}
\definecolor{mauve}{rgb}{0.58,0,0.82}

\lstset{ %
	language=python,                % the language of the code
	basicstyle=\footnotesize,       % the size of the fonts that are used for the code
	numbers=left,                   % where to put the line-numbers
	numberstyle=\tiny\color{gray},  % the style that is used for the line-numbers
	stepnumber=1,                   % the step between two line-numbers. If it's 1, each line 
	% will be numbered
	numbersep=5pt,                  % how far the line-numbers are from the code
	backgroundcolor=\color{white},  % choose the background color. You must add \usepackage{color}
	showspaces=false,               % show spaces adding particular underscores
	showstringspaces=false,         % underline spaces within strings
	showtabs=false,                 % show tabs within strings adding particular underscores
	frame=single,                   % adds a frame around the code
	rulecolor=\color{black},        % if not set, the frame-color may be changed on line-breaks within not-black text (e.g. commens (green here))
	tabsize=2,                      % sets default tabsize to 2 spaces
	captionpos=b,                   % sets the caption-position to bottom
	breaklines=true,                % sets automatic line breaking
	breakatwhitespace=false,        % sets if automatic breaks should only happen at whitespace
	title=\lstname,                 % show the filename of files included with \lstinputlisting;
	% also try caption instead of title
	keywordstyle=\color{blue},      % keyword style
	commentstyle=\color{dkgreen},   % comment style
	stringstyle=\color{mauve},      % string literal style
	escapeinside={\%*}{*)},         % if you want to add LaTeX within your code
	morekeywords={*,...}            % if you want to add more keywords to the set
}

